% Options for packages loaded elsewhere
\PassOptionsToPackage{unicode}{hyperref}
\PassOptionsToPackage{hyphens}{url}
%
\documentclass[
]{article}
\usepackage{amsmath,amssymb}
\usepackage{iftex}
\ifPDFTeX
  \usepackage[T1]{fontenc}
  \usepackage[utf8]{inputenc}
  \usepackage{textcomp} % provide euro and other symbols
\else % if luatex or xetex
  \usepackage{unicode-math} % this also loads fontspec
  \defaultfontfeatures{Scale=MatchLowercase}
  \defaultfontfeatures[\rmfamily]{Ligatures=TeX,Scale=1}
\fi
\usepackage{lmodern}
\ifPDFTeX\else
  % xetex/luatex font selection
\fi
% Use upquote if available, for straight quotes in verbatim environments
\IfFileExists{upquote.sty}{\usepackage{upquote}}{}
\IfFileExists{microtype.sty}{% use microtype if available
  \usepackage[]{microtype}
  \UseMicrotypeSet[protrusion]{basicmath} % disable protrusion for tt fonts
}{}
\makeatletter
\@ifundefined{KOMAClassName}{% if non-KOMA class
  \IfFileExists{parskip.sty}{%
    \usepackage{parskip}
  }{% else
    \setlength{\parindent}{0pt}
    \setlength{\parskip}{6pt plus 2pt minus 1pt}}
}{% if KOMA class
  \KOMAoptions{parskip=half}}
\makeatother
\usepackage{xcolor}
\usepackage[margin=1in]{geometry}
\usepackage{color}
\usepackage{fancyvrb}
\newcommand{\VerbBar}{|}
\newcommand{\VERB}{\Verb[commandchars=\\\{\}]}
\DefineVerbatimEnvironment{Highlighting}{Verbatim}{commandchars=\\\{\}}
% Add ',fontsize=\small' for more characters per line
\usepackage{framed}
\definecolor{shadecolor}{RGB}{248,248,248}
\newenvironment{Shaded}{\begin{snugshade}}{\end{snugshade}}
\newcommand{\AlertTok}[1]{\textcolor[rgb]{0.94,0.16,0.16}{#1}}
\newcommand{\AnnotationTok}[1]{\textcolor[rgb]{0.56,0.35,0.01}{\textbf{\textit{#1}}}}
\newcommand{\AttributeTok}[1]{\textcolor[rgb]{0.13,0.29,0.53}{#1}}
\newcommand{\BaseNTok}[1]{\textcolor[rgb]{0.00,0.00,0.81}{#1}}
\newcommand{\BuiltInTok}[1]{#1}
\newcommand{\CharTok}[1]{\textcolor[rgb]{0.31,0.60,0.02}{#1}}
\newcommand{\CommentTok}[1]{\textcolor[rgb]{0.56,0.35,0.01}{\textit{#1}}}
\newcommand{\CommentVarTok}[1]{\textcolor[rgb]{0.56,0.35,0.01}{\textbf{\textit{#1}}}}
\newcommand{\ConstantTok}[1]{\textcolor[rgb]{0.56,0.35,0.01}{#1}}
\newcommand{\ControlFlowTok}[1]{\textcolor[rgb]{0.13,0.29,0.53}{\textbf{#1}}}
\newcommand{\DataTypeTok}[1]{\textcolor[rgb]{0.13,0.29,0.53}{#1}}
\newcommand{\DecValTok}[1]{\textcolor[rgb]{0.00,0.00,0.81}{#1}}
\newcommand{\DocumentationTok}[1]{\textcolor[rgb]{0.56,0.35,0.01}{\textbf{\textit{#1}}}}
\newcommand{\ErrorTok}[1]{\textcolor[rgb]{0.64,0.00,0.00}{\textbf{#1}}}
\newcommand{\ExtensionTok}[1]{#1}
\newcommand{\FloatTok}[1]{\textcolor[rgb]{0.00,0.00,0.81}{#1}}
\newcommand{\FunctionTok}[1]{\textcolor[rgb]{0.13,0.29,0.53}{\textbf{#1}}}
\newcommand{\ImportTok}[1]{#1}
\newcommand{\InformationTok}[1]{\textcolor[rgb]{0.56,0.35,0.01}{\textbf{\textit{#1}}}}
\newcommand{\KeywordTok}[1]{\textcolor[rgb]{0.13,0.29,0.53}{\textbf{#1}}}
\newcommand{\NormalTok}[1]{#1}
\newcommand{\OperatorTok}[1]{\textcolor[rgb]{0.81,0.36,0.00}{\textbf{#1}}}
\newcommand{\OtherTok}[1]{\textcolor[rgb]{0.56,0.35,0.01}{#1}}
\newcommand{\PreprocessorTok}[1]{\textcolor[rgb]{0.56,0.35,0.01}{\textit{#1}}}
\newcommand{\RegionMarkerTok}[1]{#1}
\newcommand{\SpecialCharTok}[1]{\textcolor[rgb]{0.81,0.36,0.00}{\textbf{#1}}}
\newcommand{\SpecialStringTok}[1]{\textcolor[rgb]{0.31,0.60,0.02}{#1}}
\newcommand{\StringTok}[1]{\textcolor[rgb]{0.31,0.60,0.02}{#1}}
\newcommand{\VariableTok}[1]{\textcolor[rgb]{0.00,0.00,0.00}{#1}}
\newcommand{\VerbatimStringTok}[1]{\textcolor[rgb]{0.31,0.60,0.02}{#1}}
\newcommand{\WarningTok}[1]{\textcolor[rgb]{0.56,0.35,0.01}{\textbf{\textit{#1}}}}
\usepackage{graphicx}
\makeatletter
\def\maxwidth{\ifdim\Gin@nat@width>\linewidth\linewidth\else\Gin@nat@width\fi}
\def\maxheight{\ifdim\Gin@nat@height>\textheight\textheight\else\Gin@nat@height\fi}
\makeatother
% Scale images if necessary, so that they will not overflow the page
% margins by default, and it is still possible to overwrite the defaults
% using explicit options in \includegraphics[width, height, ...]{}
\setkeys{Gin}{width=\maxwidth,height=\maxheight,keepaspectratio}
% Set default figure placement to htbp
\makeatletter
\def\fps@figure{htbp}
\makeatother
\setlength{\emergencystretch}{3em} % prevent overfull lines
\providecommand{\tightlist}{%
  \setlength{\itemsep}{0pt}\setlength{\parskip}{0pt}}
\setcounter{secnumdepth}{-\maxdimen} % remove section numbering
\ifLuaTeX
  \usepackage{selnolig}  % disable illegal ligatures
\fi
\IfFileExists{bookmark.sty}{\usepackage{bookmark}}{\usepackage{hyperref}}
\IfFileExists{xurl.sty}{\usepackage{xurl}}{} % add URL line breaks if available
\urlstyle{same}
\hypersetup{
  pdftitle={Data 624 Homework 6},
  pdfauthor={Warner Alexis},
  hidelinks,
  pdfcreator={LaTeX via pandoc}}

\title{Data 624 Homework 6}
\author{Warner Alexis}
\date{2025-03-23}

\begin{document}
\maketitle

\hypertarget{arima-models}{%
\subsection{ARIMA MODELS}\label{arima-models}}

\textbf{Excercise 9.1}

Figure 9.32 shows the ACFs for 36 random numbers, 360 random numbers and
1,000 random numbers.

Explain the differences among these figures. Do they all indicate that
the data are white noise?

Let's go through the questions one by one based on \textbf{Figure 9.32}.

\begin{center}\rule{0.5\linewidth}{0.5pt}\end{center}

\hypertarget{a.-explain-the-differences-among-these-figures.-do-they-all-indicate-that-the-data-are-white-noise}{%
\subsubsection{\texorpdfstring{\textbf{1a. Explain the differences among
these figures. Do they all indicate that the data are white
noise?}}{1a. Explain the differences among these figures. Do they all indicate that the data are white noise?}}\label{a.-explain-the-differences-among-these-figures.-do-they-all-indicate-that-the-data-are-white-noise}}

\textbf{Differences among the figures:} - The \textbf{left graph (n =
36)} shows a lot of variation in the autocorrelations (ACFs), with
several spikes outside the blue significance bounds. - The
\textbf{middle graph (n = 360)} shows less variability, and almost all
autocorrelation values fall within the blue bounds. - The \textbf{right
graph (n = 1,000)} shows even less variability and very tight confidence
bands; the ACF values stay very close to zero.

\textbf{Do they all indicate white noise?} - Yes, all three plots
represent white noise series. The differences in appearance are due to
sample size. - In the left graph, the smaller sample size (n=36) causes
more random variation, which results in some autocorrelation values
falling outside the confidence bounds just by chance. - As the sample
size increases (middle and right plots), the estimates of the
autocorrelations become more precise, and the values stay within the
expected range for white noise.

\begin{center}\rule{0.5\linewidth}{0.5pt}\end{center}

\hypertarget{b.-why-are-the-critical-values-at-different-distances-from-the-mean-of-zero-why-are-the-autocorrelations-different-in-each-figure-when-they-each-refer-to-white-noise}{%
\subsubsection{\texorpdfstring{\textbf{1b. Why are the critical values
at different distances from the mean of zero? Why are the
autocorrelations different in each figure when they each refer to white
noise?}}{1b. Why are the critical values at different distances from the mean of zero? Why are the autocorrelations different in each figure when they each refer to white noise?}}\label{b.-why-are-the-critical-values-at-different-distances-from-the-mean-of-zero-why-are-the-autocorrelations-different-in-each-figure-when-they-each-refer-to-white-noise}}

\textbf{Critical values and sample size:} - The blue dashed lines
represent the 95\% confidence interval, and for white noise, they are
approximately ±1.96/√n. - As the sample size increases, the critical
values get closer to zero because √n increases, reducing the standard
error of the autocorrelation estimate. - For n = 36: ±1.96/√36 ≈ ±0.33 -
For n = 360: ±1.96/√360 ≈ ±0.10 - For n = 1,000: ±1.96/√1000 ≈ ±0.06

\textbf{Autocorrelations are different in each figure because:} - With
smaller samples, random fluctuations appear more prominently, making it
look like there's more autocorrelation. - Larger samples provide more
accurate estimates of the true autocorrelation (which is zero for white
noise), so the plots for larger n show values closer to zero.

\textbf{Excercise 9.2} A classic example of a non-stationary series are
stock prices. Plot the daily closing prices for Amazon stock (contained
in gafa\_stock), along with the ACF and PACF. Explain how each plot
shows that the series is non-stationary and should be differenced.

The time series plot of Amazon's daily closing prices reveals a clear
upward trend over time, indicating that the series is non-stationary.
This trend suggests that both the mean and variance are not constant,
which violates the assumptions of stationarity. The ACF plot further
supports this conclusion, showing a slow decay and high correlations
across many lags---behavior typical of non-stationary data. In a
stationary series, the ACF would drop off quickly after a few lags.
Similarly, the PACF plot does not exhibit a sharp cutoff, but instead
displays significant partial autocorrelations over several lags,
reinforcing the need to difference the series to achieve stationarity.

\textbf{LOading Libraries}

\begin{Shaded}
\begin{Highlighting}[]
\FunctionTok{library}\NormalTok{(latex2exp)}
\FunctionTok{library}\NormalTok{(fpp3)}
\end{Highlighting}
\end{Shaded}

\begin{verbatim}
## Registered S3 method overwritten by 'tsibble':
##   method               from 
##   as_tibble.grouped_df dplyr
\end{verbatim}

\begin{verbatim}
## -- Attaching packages -------------------------------------------- fpp3 1.0.1 --
\end{verbatim}

\begin{verbatim}
## v tibble      3.2.1     v tsibble     1.1.6
## v dplyr       1.1.4     v tsibbledata 0.4.1
## v tidyr       1.3.1     v feasts      0.4.1
## v lubridate   1.9.3     v fable       0.4.1
## v ggplot2     3.5.1
\end{verbatim}

\begin{verbatim}
## -- Conflicts ------------------------------------------------- fpp3_conflicts --
## x lubridate::date()    masks base::date()
## x dplyr::filter()      masks stats::filter()
## x tsibble::intersect() masks base::intersect()
## x tsibble::interval()  masks lubridate::interval()
## x dplyr::lag()         masks stats::lag()
## x tsibble::setdiff()   masks base::setdiff()
## x tsibble::union()     masks base::union()
\end{verbatim}

\begin{Shaded}
\begin{Highlighting}[]
\FunctionTok{library}\NormalTok{(fable)}
\FunctionTok{library}\NormalTok{(tsibble)}
\FunctionTok{library}\NormalTok{(tsibbledata)}
\end{Highlighting}
\end{Shaded}

\begin{Shaded}
\begin{Highlighting}[]
\FunctionTok{library}\NormalTok{(latex2exp)}
\NormalTok{gafa\_stock }\SpecialCharTok{|\textgreater{}} \FunctionTok{filter}\NormalTok{(Symbol }\SpecialCharTok{==} \StringTok{\textquotesingle{}AMZN\textquotesingle{}}\NormalTok{) }\SpecialCharTok{|\textgreater{}}
  \FunctionTok{gg\_tsdisplay}\NormalTok{(Close, }\AttributeTok{plot\_type =}  \StringTok{\textquotesingle{}partial\textquotesingle{}}\NormalTok{) }\SpecialCharTok{+} 
  \FunctionTok{labs}\NormalTok{(}\AttributeTok{title =} \StringTok{\textquotesingle{}Amazon Closing Stock Prize \textquotesingle{}}\NormalTok{ )}
\end{Highlighting}
\end{Shaded}

\includegraphics{Data-6224-Homework-6_files/figure-latex/unnamed-chunk-2-1.pdf}

\textbf{Excercise 9.3} For the following series, find an appropriate
Box-Cox transformation and order of differencing in order to obtain
stationary data.

\begin{itemize}
\tightlist
\item
  Turkish GDP from global\_economy.
\item
  Accommodation takings in the state of Tasmania from
  aus\_accommodation.
\item
  Monthly sales from souvenirs.
\end{itemize}

The first set of plots shows the non-transformed Turkish GDP, which
displays a strong upward trend over time, indicating clear
non-stationarity. This is supported by the autocorrelation function
(ACF) plot, which exhibits a slow and steady decay, and the partial
autocorrelation function (PACF), which shows a large spike at lag 1
followed by smaller significant lags---both of which are typical signs
of a non-stationary time series. In contrast, the second set of plots
presents the Turkish GDP after applying a Box-Cox transformation with λ
= 0.16 and first-order differencing. The transformed and differenced
series no longer shows a trend and appears to fluctuate around a
constant mean. Additionally, both the ACF and PACF now show values
mostly within the significance bounds and no clear pattern, indicating
that the series has achieved stationarity and is now suitable for time
series modeling, such as ARIMA.

\begin{Shaded}
\begin{Highlighting}[]
\NormalTok{turkey\_gdp }\OtherTok{\textless{}{-}} \CommentTok{\# plot}
\NormalTok{  global\_economy }\SpecialCharTok{\%\textgreater{}\%}
  \FunctionTok{filter}\NormalTok{(Country }\SpecialCharTok{==} \StringTok{"Turkey"}\NormalTok{)}
\CommentTok{\# plot}
\NormalTok{turkey\_gdp }\SpecialCharTok{\%\textgreater{}\%}
  \FunctionTok{gg\_tsdisplay}\NormalTok{(GDP, }\AttributeTok{plot\_type=}\StringTok{\textquotesingle{}partial\textquotesingle{}}\NormalTok{) }\SpecialCharTok{+}
  \FunctionTok{labs}\NormalTok{(}\AttributeTok{title =} \StringTok{"Non{-}transformed Turkish GDP"}\NormalTok{)}
\end{Highlighting}
\end{Shaded}

\includegraphics{Data-6224-Homework-6_files/figure-latex/unnamed-chunk-3-1.pdf}

\begin{Shaded}
\begin{Highlighting}[]
\CommentTok{\# calculate lambda}
\NormalTok{lambda }\OtherTok{\textless{}{-}}\NormalTok{ turkey\_gdp }\SpecialCharTok{\%\textgreater{}\%}
  \FunctionTok{features}\NormalTok{(GDP, }\AttributeTok{features =}\NormalTok{ guerrero) }\SpecialCharTok{\%\textgreater{}\%}
  \FunctionTok{pull}\NormalTok{(lambda\_guerrero)}


\NormalTok{turkey\_gdp }\SpecialCharTok{\%\textgreater{}\%}
  \FunctionTok{features}\NormalTok{(}\FunctionTok{box\_cox}\NormalTok{(GDP,lambda), unitroot\_ndiffs) }
\end{Highlighting}
\end{Shaded}

\begin{verbatim}
## # A tibble: 1 x 2
##   Country ndiffs
##   <fct>    <int>
## 1 Turkey       1
\end{verbatim}

\begin{Shaded}
\begin{Highlighting}[]
\CommentTok{\# unit root test}


\CommentTok{\# transformed plot}
\NormalTok{turkey\_gdp }\SpecialCharTok{\%\textgreater{}\%}
  \FunctionTok{gg\_tsdisplay}\NormalTok{(}\FunctionTok{difference}\NormalTok{(}\FunctionTok{box\_cox}\NormalTok{(GDP,lambda)), }\AttributeTok{plot\_type=}\StringTok{\textquotesingle{}partial\textquotesingle{}}\NormalTok{) }\SpecialCharTok{+}
  \FunctionTok{labs}\NormalTok{(}\AttributeTok{title =} \FunctionTok{TeX}\NormalTok{(}\FunctionTok{paste0}\NormalTok{(}\StringTok{"Differenced Turkish GDP with $}\SpecialCharTok{\textbackslash{}\textbackslash{}}\StringTok{lambda$ = "}\NormalTok{,}
                          \FunctionTok{round}\NormalTok{(lambda,}\DecValTok{2}\NormalTok{))))}
\end{Highlighting}
\end{Shaded}

\includegraphics{Data-6224-Homework-6_files/figure-latex/unnamed-chunk-3-2.pdf}

The first set of plots displays the non-transformed Tasmania
accommodation takings, which exhibit a clear upward trend and strong
seasonal pattern, both indicators of a non-stationary series. The
autocorrelation function (ACF) plot shows significant spikes at seasonal
lags (e.g., lag 4, 8, 12, etc.), indicating the presence of quarterly
seasonality, while the partial autocorrelation function (PACF) also
reflects persistent seasonal effects. The second set of plots shows the
same series after applying a Box-Cox transformation with λ = 0 (log
transformation) and seasonal differencing. The differenced and
transformed series appears more stable, with the trend removed and the
seasonal pattern largely eliminated. This is confirmed by the ACF and
PACF plots, which now show autocorrelations that fall within the
confidence bounds or decay more quickly, indicating that the series is
now approximately stationary and suitable for further time series
modeling.

\begin{Shaded}
\begin{Highlighting}[]
\NormalTok{tasma\_acc }\OtherTok{\textless{}{-}} 
\NormalTok{  aus\_accommodation }\SpecialCharTok{\%\textgreater{}\%}
  \FunctionTok{filter}\NormalTok{(State }\SpecialCharTok{==} \StringTok{"Tasmania"}\NormalTok{)}


\NormalTok{tasma\_acc }\SpecialCharTok{\%\textgreater{}\%}
  \FunctionTok{gg\_tsdisplay}\NormalTok{(Takings, }\AttributeTok{plot\_type=}\StringTok{\textquotesingle{}partial\textquotesingle{}}\NormalTok{) }\SpecialCharTok{+}
  \FunctionTok{labs}\NormalTok{(}\AttributeTok{title =} \StringTok{"Non{-}transformed Tasmania Accomodation Takings"}\NormalTok{)}
\end{Highlighting}
\end{Shaded}

\includegraphics{Data-6224-Homework-6_files/figure-latex/unnamed-chunk-4-1.pdf}

\begin{Shaded}
\begin{Highlighting}[]
\CommentTok{\# calculate lambda}
\NormalTok{lambda }\OtherTok{\textless{}{-}}\NormalTok{tasma\_acc }\SpecialCharTok{\%\textgreater{}\%}
  \FunctionTok{features}\NormalTok{(Takings, }\AttributeTok{features =}\NormalTok{ guerrero) }\SpecialCharTok{\%\textgreater{}\%}
  \FunctionTok{pull}\NormalTok{(lambda\_guerrero)}

\CommentTok{\#unit root test}
\NormalTok{tasma\_acc }\SpecialCharTok{\%\textgreater{}\%}
  \FunctionTok{features}\NormalTok{(}\FunctionTok{box\_cox}\NormalTok{(Takings,lambda), unitroot\_nsdiffs)}
\end{Highlighting}
\end{Shaded}

\begin{verbatim}
## # A tibble: 1 x 2
##   State    nsdiffs
##   <chr>      <int>
## 1 Tasmania       1
\end{verbatim}

\begin{Shaded}
\begin{Highlighting}[]
\NormalTok{tasma\_acc }\SpecialCharTok{\%\textgreater{}\%}
  \FunctionTok{gg\_tsdisplay}\NormalTok{(}\FunctionTok{difference}\NormalTok{(}\FunctionTok{box\_cox}\NormalTok{(Takings,lambda), }\DecValTok{4}\NormalTok{), }\AttributeTok{plot\_type=}\StringTok{\textquotesingle{}partial\textquotesingle{}}\NormalTok{) }\SpecialCharTok{+}
  \FunctionTok{labs}\NormalTok{(}\AttributeTok{title =} \FunctionTok{TeX}\NormalTok{(}\FunctionTok{paste0}\NormalTok{(}\StringTok{"Differenced Tasmania Accomodation Takings with $}\SpecialCharTok{\textbackslash{}\textbackslash{}}\StringTok{lambda$ = "}\NormalTok{,}
                          \FunctionTok{round}\NormalTok{(lambda,}\DecValTok{2}\NormalTok{))))}
\end{Highlighting}
\end{Shaded}

\begin{verbatim}
## Warning: Removed 4 rows containing missing values or values outside the scale range
## (`geom_line()`).
\end{verbatim}

\begin{verbatim}
## Warning: Removed 4 rows containing missing values or values outside the scale range
## (`geom_point()`).
\end{verbatim}

\includegraphics{Data-6224-Homework-6_files/figure-latex/unnamed-chunk-4-2.pdf}

\begin{Shaded}
\begin{Highlighting}[]
\CommentTok{\# plot}
\NormalTok{souvenirs }\SpecialCharTok{\%\textgreater{}\%}
  \FunctionTok{gg\_tsdisplay}\NormalTok{(Sales, }\AttributeTok{plot\_type=}\StringTok{\textquotesingle{}partial\textquotesingle{}}\NormalTok{, }\AttributeTok{lag =} \DecValTok{36}\NormalTok{) }\SpecialCharTok{+}
  \FunctionTok{labs}\NormalTok{(}\AttributeTok{title =} \StringTok{"Non{-}transformed Monthly Souvenir Sales"}\NormalTok{)}
\end{Highlighting}
\end{Shaded}

\includegraphics{Data-6224-Homework-6_files/figure-latex/unnamed-chunk-5-1.pdf}

\begin{Shaded}
\begin{Highlighting}[]
\CommentTok{\# calculate lambda}
\NormalTok{lambda }\OtherTok{\textless{}{-}}\NormalTok{ souvenirs }\SpecialCharTok{\%\textgreater{}\%}
  \FunctionTok{features}\NormalTok{(Sales, }\AttributeTok{features =}\NormalTok{ guerrero) }\SpecialCharTok{\%\textgreater{}\%}
  \FunctionTok{pull}\NormalTok{(lambda\_guerrero)}

\CommentTok{\# unit root test}
\NormalTok{souvenirs }\SpecialCharTok{\%\textgreater{}\%}
  \FunctionTok{features}\NormalTok{(}\FunctionTok{box\_cox}\NormalTok{(Sales,lambda), unitroot\_nsdiffs)}
\end{Highlighting}
\end{Shaded}

\begin{verbatim}
## # A tibble: 1 x 1
##   nsdiffs
##     <int>
## 1       1
\end{verbatim}

\begin{Shaded}
\begin{Highlighting}[]
\NormalTok{souvenirs }\SpecialCharTok{\%\textgreater{}\%}
  \FunctionTok{gg\_tsdisplay}\NormalTok{(}\FunctionTok{difference}\NormalTok{(}\FunctionTok{box\_cox}\NormalTok{(Sales,lambda), }\DecValTok{12}\NormalTok{), }\AttributeTok{plot\_type=}\StringTok{\textquotesingle{}partial\textquotesingle{}}\NormalTok{, }\AttributeTok{lag =} \DecValTok{36}\NormalTok{) }\SpecialCharTok{+}
  \FunctionTok{labs}\NormalTok{(}\AttributeTok{title =} \FunctionTok{TeX}\NormalTok{(}\FunctionTok{paste0}\NormalTok{(}\StringTok{"Differenced Monthly Souvenir Sales with $}\SpecialCharTok{\textbackslash{}\textbackslash{}}\StringTok{lambda$ = "}\NormalTok{,}
                          \FunctionTok{round}\NormalTok{(lambda,}\DecValTok{2}\NormalTok{))))}
\end{Highlighting}
\end{Shaded}

\includegraphics{Data-6224-Homework-6_files/figure-latex/unnamed-chunk-5-2.pdf}

\textbf{Excercise 9.5} For your retail data (from Exercise 7 in Section
2.10), find the appropriate order of differencing (after transformation
if necessary) to obtain stationary data.

The first set of plots presents the non-transformed retail turnover data
for Tasmania, which displays a strong upward trend and increasing
variance over time---both signs of non-stationarity. The ACF shows very
slow decay, with significant autocorrelations extending across many
lags, while the PACF shows a large spike at lag 1 followed by smaller
but still significant values. These patterns confirm that the series is
non-stationary in both mean and variance. In the second set of plots,
the data have been transformed using a Box-Cox transformation with λ =
0.27 and differenced once. The resulting series appears more stable,
with the trend largely removed and the variance stabilized. The ACF now
shows a much quicker decay, and the PACF displays a reduced number of
significant lags. Although some seasonality may remain, the primary
trend has been addressed.

\begin{itemize}
\tightlist
\item
  \textbf{Box-Cox transformation} with \textbf{λ = 0.27} is appropriate
  to stabilize variance.
\item
  \textbf{First-order differencing} is sufficient to remove the trend
  and achieve approximate stationarity.
\item
  Additional \textbf{seasonal differencing} (e.g., lag 12) could be
  explored if residual seasonality is detected during modeling.
\end{itemize}

\begin{Shaded}
\begin{Highlighting}[]
\FunctionTok{set.seed}\NormalTok{(}\DecValTok{000}\NormalTok{)}
\NormalTok{myseries }\OtherTok{\textless{}{-}}\NormalTok{ aus\_retail }\SpecialCharTok{\%\textgreater{}\%}
  \FunctionTok{filter}\NormalTok{(}\StringTok{\textasciigrave{}}\AttributeTok{Series ID}\StringTok{\textasciigrave{}} \SpecialCharTok{==} \FunctionTok{sample}\NormalTok{(aus\_retail}\SpecialCharTok{$}\StringTok{\textasciigrave{}}\AttributeTok{Series ID}\StringTok{\textasciigrave{}}\NormalTok{,}\DecValTok{1}\NormalTok{)) }

\CommentTok{\# plot}
\NormalTok{myseries }\SpecialCharTok{\%\textgreater{}\%}
  \FunctionTok{gg\_tsdisplay}\NormalTok{(Turnover, }\AttributeTok{plot\_type=}\StringTok{\textquotesingle{}partial\textquotesingle{}}\NormalTok{, }\AttributeTok{lag =} \DecValTok{36}\NormalTok{) }\SpecialCharTok{+}
  \FunctionTok{labs}\NormalTok{(}\AttributeTok{title =} \StringTok{"Non{-}transformed Retail Turnover"}\NormalTok{)}
\end{Highlighting}
\end{Shaded}

\includegraphics{Data-6224-Homework-6_files/figure-latex/unnamed-chunk-6-1.pdf}

\begin{Shaded}
\begin{Highlighting}[]
\CommentTok{\# lambda calculation}
\NormalTok{lambda }\OtherTok{\textless{}{-}}\NormalTok{ myseries }\SpecialCharTok{\%\textgreater{}\%}
  \FunctionTok{features}\NormalTok{(Turnover, }\AttributeTok{features =}\NormalTok{ guerrero) }\SpecialCharTok{\%\textgreater{}\%}
  \FunctionTok{pull}\NormalTok{(lambda\_guerrero)}

\CommentTok{\# unit root test}
\NormalTok{myseries }\SpecialCharTok{\%\textgreater{}\%}
  \FunctionTok{features}\NormalTok{(}\FunctionTok{box\_cox}\NormalTok{(Turnover, lambda), unitroot\_nsdiffs) }
\end{Highlighting}
\end{Shaded}

\begin{verbatim}
## # A tibble: 1 x 3
##   State           Industry               nsdiffs
##   <chr>           <chr>                    <int>
## 1 New South Wales Takeaway food services       1
\end{verbatim}

\begin{Shaded}
\begin{Highlighting}[]
\NormalTok{myseries }\SpecialCharTok{\%\textgreater{}\%}
  \FunctionTok{gg\_tsdisplay}\NormalTok{(}\FunctionTok{difference}\NormalTok{(}\FunctionTok{box\_cox}\NormalTok{(Turnover,lambda), }\DecValTok{12}\NormalTok{), }\AttributeTok{plot\_type=}\StringTok{\textquotesingle{}partial\textquotesingle{}}\NormalTok{, }\AttributeTok{lag =} \DecValTok{36}\NormalTok{) }\SpecialCharTok{+}
  \FunctionTok{labs}\NormalTok{(}\AttributeTok{title =} \FunctionTok{TeX}\NormalTok{(}\FunctionTok{paste0}\NormalTok{(}\StringTok{"Differenced Tasmania Accomodation Takings with $}\SpecialCharTok{\textbackslash{}\textbackslash{}}\StringTok{lambda$ = "}\NormalTok{,}
                          \FunctionTok{round}\NormalTok{(lambda,}\DecValTok{2}\NormalTok{))))}
\end{Highlighting}
\end{Shaded}

\includegraphics{Data-6224-Homework-6_files/figure-latex/unnamed-chunk-6-2.pdf}

\textbf{Excercise 9.6}

Simulate and plot some data from simple ARIMA models.

Use the following R code to generate data from an AR(1) model with

This exercise explores the behavior of various ARIMA models through
simulation and visualization. Starting with an AR(1) model with ϕ₁ =
0.6, the series shows moderate persistence, and increasing ϕ₁ leads to
smoother, more correlated patterns, while lower values make the series
resemble white noise. The MA(1) model with θ₁ = 0.6 generates a series
influenced by past shocks, though the effect dissipates quickly compared
to AR models. Combining both effects, the ARMA(1,1) model demonstrates
smoother yet still responsive behavior, balancing the autocorrelation
from AR and the noise filtering of MA. A non-stationary AR(2) model with
ϕ₁ = -0.8 and ϕ₂ = 0.3 produces a series with irregular and unstable
dynamics, often with large swings and no consistent mean. Comparing the
ARMA(1,1) and AR(2) plots highlights the contrast between a stable,
stationary process and one that lacks mean reversion, emphasizing the
importance of parameter selection in time series modeling.

\begin{Shaded}
\begin{Highlighting}[]
\CommentTok{\# a }
\NormalTok{y }\OtherTok{\textless{}{-}} \FunctionTok{numeric}\NormalTok{(}\DecValTok{100}\NormalTok{)}
\NormalTok{e }\OtherTok{\textless{}{-}} \FunctionTok{rnorm}\NormalTok{(}\DecValTok{100}\NormalTok{)}
\ControlFlowTok{for}\NormalTok{(i }\ControlFlowTok{in} \DecValTok{2}\SpecialCharTok{:}\DecValTok{100}\NormalTok{) \{}
\NormalTok{  y[i] }\OtherTok{\textless{}{-}} \FloatTok{0.6} \SpecialCharTok{*}\NormalTok{ y[i }\SpecialCharTok{{-}} \DecValTok{1}\NormalTok{] }\SpecialCharTok{+}\NormalTok{ e[i]}
\NormalTok{\}}
\NormalTok{sim }\OtherTok{\textless{}{-}} \FunctionTok{tsibble}\NormalTok{(}\AttributeTok{idx =} \FunctionTok{seq\_len}\NormalTok{(}\DecValTok{100}\NormalTok{), }\AttributeTok{y =}\NormalTok{ y, }\AttributeTok{index =}\NormalTok{ idx)}
\CommentTok{\# b }
\NormalTok{sim }\SpecialCharTok{\%\textgreater{}\%}
  \FunctionTok{autoplot}\NormalTok{(y) }\SpecialCharTok{+}
  \FunctionTok{labs}\NormalTok{(}\AttributeTok{title =} \StringTok{"AR(1) Time Series with ϕ₁ = 0.6"}\NormalTok{, }\AttributeTok{y =} \StringTok{"y"}\NormalTok{)}
\end{Highlighting}
\end{Shaded}

\begin{verbatim}
## Warning in grid.Call(C_textBounds, as.graphicsAnnot(x$label), x$x, x$y, :
## conversion failure on 'AR(1) Time Series with ϕ₁ = 0.6' in 'mbcsToSbcs': dot
## substituted for <cf>
\end{verbatim}

\begin{verbatim}
## Warning in grid.Call(C_textBounds, as.graphicsAnnot(x$label), x$x, x$y, :
## conversion failure on 'AR(1) Time Series with ϕ₁ = 0.6' in 'mbcsToSbcs': dot
## substituted for <95>
\end{verbatim}

\begin{verbatim}
## Warning in grid.Call.graphics(C_text, as.graphicsAnnot(x$label), x$x, x$y, :
## conversion failure on 'AR(1) Time Series with ϕ₁ = 0.6' in 'mbcsToSbcs': dot
## substituted for <cf>
\end{verbatim}

\begin{verbatim}
## Warning in grid.Call.graphics(C_text, as.graphicsAnnot(x$label), x$x, x$y, :
## conversion failure on 'AR(1) Time Series with ϕ₁ = 0.6' in 'mbcsToSbcs': dot
## substituted for <95>
\end{verbatim}

\includegraphics{Data-6224-Homework-6_files/figure-latex/unnamed-chunk-7-1.pdf}

\begin{Shaded}
\begin{Highlighting}[]
\CommentTok{\# c }
\NormalTok{e }\OtherTok{\textless{}{-}} \FunctionTok{rnorm}\NormalTok{(}\DecValTok{101}\NormalTok{)}
\NormalTok{y }\OtherTok{\textless{}{-}} \FunctionTok{numeric}\NormalTok{(}\DecValTok{100}\NormalTok{)}
\ControlFlowTok{for}\NormalTok{(i }\ControlFlowTok{in} \DecValTok{1}\SpecialCharTok{:}\DecValTok{100}\NormalTok{)\{}
\NormalTok{  y[i] }\OtherTok{\textless{}{-}}\NormalTok{ e[i}\SpecialCharTok{+}\DecValTok{1}\NormalTok{] }\SpecialCharTok{+} \FloatTok{0.6} \SpecialCharTok{*}\NormalTok{ e[i]}
\NormalTok{\}}
\NormalTok{sim\_ma }\OtherTok{\textless{}{-}} \FunctionTok{tsibble}\NormalTok{(}\AttributeTok{idx =} \FunctionTok{seq\_len}\NormalTok{(}\DecValTok{100}\NormalTok{), }\AttributeTok{y =}\NormalTok{ y, }\AttributeTok{index =}\NormalTok{ idx)}

\CommentTok{\#d}
\NormalTok{sim\_ma }\SpecialCharTok{\%\textgreater{}\%}
  \FunctionTok{autoplot}\NormalTok{(y) }\SpecialCharTok{+}
  \FunctionTok{labs}\NormalTok{(}\AttributeTok{title =} \StringTok{"MA(1) Time Series with θ₁ = 0.6"}\NormalTok{, }\AttributeTok{y =} \StringTok{"y"}\NormalTok{)}
\end{Highlighting}
\end{Shaded}

\begin{verbatim}
## Warning in grid.Call(C_textBounds, as.graphicsAnnot(x$label), x$x, x$y, :
## conversion failure on 'MA(1) Time Series with θ₁ = 0.6' in 'mbcsToSbcs': dot
## substituted for <ce>
\end{verbatim}

\begin{verbatim}
## Warning in grid.Call(C_textBounds, as.graphicsAnnot(x$label), x$x, x$y, :
## conversion failure on 'MA(1) Time Series with θ₁ = 0.6' in 'mbcsToSbcs': dot
## substituted for <b8>
\end{verbatim}

\begin{verbatim}
## Warning in grid.Call.graphics(C_text, as.graphicsAnnot(x$label), x$x, x$y, :
## conversion failure on 'MA(1) Time Series with θ₁ = 0.6' in 'mbcsToSbcs': dot
## substituted for <ce>
\end{verbatim}

\begin{verbatim}
## Warning in grid.Call.graphics(C_text, as.graphicsAnnot(x$label), x$x, x$y, :
## conversion failure on 'MA(1) Time Series with θ₁ = 0.6' in 'mbcsToSbcs': dot
## substituted for <b8>
\end{verbatim}

\includegraphics{Data-6224-Homework-6_files/figure-latex/unnamed-chunk-7-2.pdf}

\begin{Shaded}
\begin{Highlighting}[]
\CommentTok{\#e}
\NormalTok{e }\OtherTok{\textless{}{-}} \FunctionTok{rnorm}\NormalTok{(}\DecValTok{101}\NormalTok{)}
\NormalTok{y }\OtherTok{\textless{}{-}} \FunctionTok{numeric}\NormalTok{(}\DecValTok{100}\NormalTok{)}
\ControlFlowTok{for}\NormalTok{(i }\ControlFlowTok{in} \DecValTok{2}\SpecialCharTok{:}\DecValTok{100}\NormalTok{)\{}
\NormalTok{  y[i] }\OtherTok{\textless{}{-}} \FloatTok{0.6} \SpecialCharTok{*}\NormalTok{ y[i }\SpecialCharTok{{-}} \DecValTok{1}\NormalTok{] }\SpecialCharTok{+}\NormalTok{ e[i] }\SpecialCharTok{+} \FloatTok{0.6} \SpecialCharTok{*}\NormalTok{ e[i }\SpecialCharTok{{-}} \DecValTok{1}\NormalTok{]}
\NormalTok{\}}
\NormalTok{sim\_arma }\OtherTok{\textless{}{-}} \FunctionTok{tsibble}\NormalTok{(}\AttributeTok{idx =} \FunctionTok{seq\_len}\NormalTok{(}\DecValTok{100}\NormalTok{), }\AttributeTok{y =}\NormalTok{ y, }\AttributeTok{index =}\NormalTok{ idx)}
\FunctionTok{autoplot}\NormalTok{(sim\_arma, y) }\SpecialCharTok{+} \FunctionTok{labs}\NormalTok{(}\AttributeTok{title =} \StringTok{"ARMA(1,1) Series"}\NormalTok{, }\AttributeTok{y =} \StringTok{"y"}\NormalTok{)}
\end{Highlighting}
\end{Shaded}

\includegraphics{Data-6224-Homework-6_files/figure-latex/unnamed-chunk-7-3.pdf}

\begin{Shaded}
\begin{Highlighting}[]
\CommentTok{\# f}
\NormalTok{e }\OtherTok{\textless{}{-}} \FunctionTok{rnorm}\NormalTok{(}\DecValTok{102}\NormalTok{)}
\NormalTok{y }\OtherTok{\textless{}{-}} \FunctionTok{numeric}\NormalTok{(}\DecValTok{100}\NormalTok{)}
\ControlFlowTok{for}\NormalTok{(i }\ControlFlowTok{in} \DecValTok{3}\SpecialCharTok{:}\DecValTok{100}\NormalTok{)\{}
\NormalTok{  y[i] }\OtherTok{\textless{}{-}} \SpecialCharTok{{-}}\FloatTok{0.8} \SpecialCharTok{*}\NormalTok{ y[i }\SpecialCharTok{{-}} \DecValTok{1}\NormalTok{] }\SpecialCharTok{+} \FloatTok{0.3} \SpecialCharTok{*}\NormalTok{ y[i }\SpecialCharTok{{-}} \DecValTok{2}\NormalTok{] }\SpecialCharTok{+}\NormalTok{ e[i]}
\NormalTok{\}}
\NormalTok{sim\_ar2 }\OtherTok{\textless{}{-}} \FunctionTok{tsibble}\NormalTok{(}\AttributeTok{idx =} \FunctionTok{seq\_len}\NormalTok{(}\DecValTok{100}\NormalTok{), }\AttributeTok{y =}\NormalTok{ y, }\AttributeTok{index =}\NormalTok{ idx)}
\FunctionTok{autoplot}\NormalTok{(sim\_ar2, y) }\SpecialCharTok{+} \FunctionTok{labs}\NormalTok{(}\AttributeTok{title =} \StringTok{"AR(2) Series (Non{-}stationary)"}\NormalTok{, }\AttributeTok{y =} \StringTok{"y"}\NormalTok{)}
\end{Highlighting}
\end{Shaded}

\includegraphics{Data-6224-Homework-6_files/figure-latex/unnamed-chunk-7-4.pdf}

\textbf{Excercise 9.7}

The model can be written in term the baskshift operator

\(y_t = -0.87 \varepsilon_{t-1} + \varepsilon_t\)

\[
(1 - B)^2 y_t = (1 - 0.87B) \varepsilon_t
\]

\begin{Shaded}
\begin{Highlighting}[]
\NormalTok{fit }\OtherTok{\textless{}{-}}\NormalTok{ aus\_airpassengers }\SpecialCharTok{\%\textgreater{}\%}
  \FunctionTok{filter}\NormalTok{(Year }\SpecialCharTok{\textless{}} \DecValTok{2012}\NormalTok{) }\SpecialCharTok{\%\textgreater{}\%}
  \FunctionTok{model}\NormalTok{(}\FunctionTok{ARIMA}\NormalTok{(Passengers)) }

\FunctionTok{report}\NormalTok{(fit)}
\end{Highlighting}
\end{Shaded}

\begin{verbatim}
## Series: Passengers 
## Model: ARIMA(0,2,1) 
## 
## Coefficients:
##           ma1
##       -0.8756
## s.e.   0.0722
## 
## sigma^2 estimated as 4.671:  log likelihood=-87.8
## AIC=179.61   AICc=179.93   BIC=182.99
\end{verbatim}

\begin{Shaded}
\begin{Highlighting}[]
\NormalTok{fit }\SpecialCharTok{\%\textgreater{}\%} 
  \FunctionTok{forecast}\NormalTok{(}\AttributeTok{h=}\DecValTok{10}\NormalTok{) }\SpecialCharTok{\%\textgreater{}\%}
  \FunctionTok{autoplot}\NormalTok{(aus\_airpassengers) }\SpecialCharTok{+}
  \FunctionTok{labs}\NormalTok{(}\AttributeTok{title =} \StringTok{"Australian Aircraft Passengers with ARIMA(0,2,1)"}\NormalTok{, }\AttributeTok{y =} \StringTok{"Passengers (in millions)"}\NormalTok{)}
\end{Highlighting}
\end{Shaded}

\includegraphics{Data-6224-Homework-6_files/figure-latex/unnamed-chunk-8-1.pdf}

\begin{Shaded}
\begin{Highlighting}[]
\NormalTok{fit }\SpecialCharTok{\%\textgreater{}\%} 
  \FunctionTok{gg\_tsresiduals}\NormalTok{() }\SpecialCharTok{+}
  \FunctionTok{labs}\NormalTok{(}\AttributeTok{title =} \StringTok{"Residuals for ARIMA(0,2,1)"}\NormalTok{)}
\end{Highlighting}
\end{Shaded}

\includegraphics{Data-6224-Homework-6_files/figure-latex/unnamed-chunk-8-2.pdf}

\begin{itemize}
\item
  \begin{enumerate}
  \def\labelenumi{\alph{enumi}.}
  \setcounter{enumi}{2}
  \tightlist
  \item
    Plot forecasts from an ARIMA(0,1,0) model with drift and compare
    these to part a.
  \end{enumerate}
\end{itemize}

The ARIMA model from part (a) forecasted values that were higher than
the actual observations, whereas this ARIMA model predicted values that
were lower than the actual outcomes. Additionally, the slope of the
forecasts appears to be more gradual, indicating a slower rate of change
over time.

\begin{Shaded}
\begin{Highlighting}[]
\NormalTok{fit2 }\OtherTok{\textless{}{-}}\NormalTok{aus\_airpassengers }\SpecialCharTok{\%\textgreater{}\%}
  \FunctionTok{filter}\NormalTok{(Year }\SpecialCharTok{\textless{}} \DecValTok{2012}\NormalTok{) }\SpecialCharTok{\%\textgreater{}\%}
  \FunctionTok{model}\NormalTok{(}\FunctionTok{ARIMA}\NormalTok{(Passengers }\SpecialCharTok{\textasciitilde{}} \FunctionTok{pdq}\NormalTok{(}\DecValTok{0}\NormalTok{,}\DecValTok{1}\NormalTok{,}\DecValTok{0}\NormalTok{)))}

\NormalTok{fit2 }\SpecialCharTok{\%\textgreater{}\%} 
  \FunctionTok{forecast}\NormalTok{(}\AttributeTok{h=}\DecValTok{10}\NormalTok{) }\SpecialCharTok{\%\textgreater{}\%}
  \FunctionTok{autoplot}\NormalTok{(aus\_airpassengers) }\SpecialCharTok{+}
  \FunctionTok{labs}\NormalTok{(}\AttributeTok{title =} \StringTok{"Australian Aircraft Passengers with ARIMA(0,1,0)"}\NormalTok{, }\AttributeTok{y =} \StringTok{"Passengers (in millions)"}\NormalTok{)}
\end{Highlighting}
\end{Shaded}

\includegraphics{Data-6224-Homework-6_files/figure-latex/unnamed-chunk-9-1.pdf}

\begin{Shaded}
\begin{Highlighting}[]
\NormalTok{fit2 }\SpecialCharTok{\%\textgreater{}\%} 
  \FunctionTok{gg\_tsresiduals}\NormalTok{() }\SpecialCharTok{+}
  \FunctionTok{labs}\NormalTok{(}\AttributeTok{title =} \StringTok{"Residuals for ARIMA(0,1,0)"}\NormalTok{)}
\end{Highlighting}
\end{Shaded}

\includegraphics{Data-6224-Homework-6_files/figure-latex/unnamed-chunk-9-2.pdf}
- d.~Plot forecasts from an ARIMA(2,1,2) model with drift and compare
these to parts a and c.~Remove the constant and see what happens

It is more similar to part (a), and the residuals appear to resemble
white noise. When the constant is removed, the model effectively becomes
a null model. Omitting the constant results in the forecast including a
polynomial trend of order \(d - 1\)---which, in this case, is
0---rendering the series non-stationary.

\begin{Shaded}
\begin{Highlighting}[]
\NormalTok{fit3 }\OtherTok{\textless{}{-}}\NormalTok{aus\_airpassengers }\SpecialCharTok{\%\textgreater{}\%}
  \FunctionTok{filter}\NormalTok{(Year }\SpecialCharTok{\textless{}} \DecValTok{2012}\NormalTok{) }\SpecialCharTok{\%\textgreater{}\%}
  \FunctionTok{model}\NormalTok{(}\FunctionTok{ARIMA}\NormalTok{(Passengers }\SpecialCharTok{\textasciitilde{}} \FunctionTok{pdq}\NormalTok{(}\DecValTok{2}\NormalTok{,}\DecValTok{1}\NormalTok{,}\DecValTok{2}\NormalTok{)))}

\NormalTok{fit3 }\SpecialCharTok{\%\textgreater{}\%} 
  \FunctionTok{forecast}\NormalTok{(}\AttributeTok{h=}\DecValTok{10}\NormalTok{) }\SpecialCharTok{\%\textgreater{}\%}
  \FunctionTok{autoplot}\NormalTok{(aus\_airpassengers) }\SpecialCharTok{+}
  \FunctionTok{labs}\NormalTok{(}\AttributeTok{title =} \StringTok{"Australian Aircraft Passengers with ARIMA(2,1,2)"}\NormalTok{, }\AttributeTok{y =} \StringTok{"Passengers (in millions)"}\NormalTok{)}
\end{Highlighting}
\end{Shaded}

\includegraphics{Data-6224-Homework-6_files/figure-latex/unnamed-chunk-10-1.pdf}

\begin{Shaded}
\begin{Highlighting}[]
\NormalTok{fit3 }\SpecialCharTok{\%\textgreater{}\%} 
  \FunctionTok{gg\_tsresiduals}\NormalTok{() }\SpecialCharTok{+}
  \FunctionTok{labs}\NormalTok{(}\AttributeTok{title =} \StringTok{"Residuals for ARIMA(2,1,2)"}\NormalTok{)}
\end{Highlighting}
\end{Shaded}

\includegraphics{Data-6224-Homework-6_files/figure-latex/unnamed-chunk-10-2.pdf}

\begin{Shaded}
\begin{Highlighting}[]
\CommentTok{\#removing constant}
\NormalTok{fit4 }\OtherTok{\textless{}{-}}\NormalTok{aus\_airpassengers }\SpecialCharTok{\%\textgreater{}\%}
  \FunctionTok{filter}\NormalTok{(Year }\SpecialCharTok{\textless{}} \DecValTok{2012}\NormalTok{) }\SpecialCharTok{\%\textgreater{}\%}
  \FunctionTok{model}\NormalTok{(}\FunctionTok{ARIMA}\NormalTok{(Passengers }\SpecialCharTok{\textasciitilde{}} \DecValTok{0} \SpecialCharTok{+} \FunctionTok{pdq}\NormalTok{(}\DecValTok{2}\NormalTok{,}\DecValTok{1}\NormalTok{,}\DecValTok{2}\NormalTok{)))}

\FunctionTok{report}\NormalTok{(fit4)}
\end{Highlighting}
\end{Shaded}

\begin{verbatim}
## Series: Passengers 
## Model: NULL model 
## NULL model
\end{verbatim}

\begin{enumerate}
\def\labelenumi{\alph{enumi}.}
\setcounter{enumi}{4}
\tightlist
\item
  Plot forecasts from an ARIMA(0,2,1) model with a constant. What
  happens?
\end{enumerate}

\begin{Shaded}
\begin{Highlighting}[]
\NormalTok{fit5 }\OtherTok{\textless{}{-}}\NormalTok{aus\_airpassengers }\SpecialCharTok{\%\textgreater{}\%}
  \FunctionTok{filter}\NormalTok{(Year }\SpecialCharTok{\textless{}} \DecValTok{2012}\NormalTok{) }\SpecialCharTok{\%\textgreater{}\%}
  \FunctionTok{model}\NormalTok{(}\FunctionTok{ARIMA}\NormalTok{(Passengers }\SpecialCharTok{\textasciitilde{}} \DecValTok{1} \SpecialCharTok{+} \FunctionTok{pdq}\NormalTok{(}\DecValTok{0}\NormalTok{,}\DecValTok{2}\NormalTok{,}\DecValTok{1}\NormalTok{)))}
\end{Highlighting}
\end{Shaded}

\begin{verbatim}
## Warning: Model specification induces a quadratic or higher order polynomial trend. 
## This is generally discouraged, consider removing the constant or reducing the number of differences.
\end{verbatim}

\begin{Shaded}
\begin{Highlighting}[]
\NormalTok{fit5 }\SpecialCharTok{\%\textgreater{}\%} 
  \FunctionTok{forecast}\NormalTok{(}\AttributeTok{h=}\DecValTok{10}\NormalTok{) }\SpecialCharTok{\%\textgreater{}\%}
  \FunctionTok{autoplot}\NormalTok{(aus\_airpassengers) }\SpecialCharTok{+}
  \FunctionTok{labs}\NormalTok{(}\AttributeTok{title =} \StringTok{"Australian Aircraft Passengers with ARIMA(0,2,1) with constant"}\NormalTok{, }\AttributeTok{y =} \StringTok{"Passengers (in millions)"}\NormalTok{)}
\end{Highlighting}
\end{Shaded}

\includegraphics{Data-6224-Homework-6_files/figure-latex/unnamed-chunk-11-1.pdf}

\begin{Shaded}
\begin{Highlighting}[]
\NormalTok{fit5 }\SpecialCharTok{\%\textgreater{}\%} 
  \FunctionTok{gg\_tsresiduals}\NormalTok{() }\SpecialCharTok{+}
  \FunctionTok{labs}\NormalTok{(}\AttributeTok{title =} \StringTok{"Residuals for ARIMA(0,2,1) "}\NormalTok{)}
\end{Highlighting}
\end{Shaded}

\includegraphics{Data-6224-Homework-6_files/figure-latex/unnamed-chunk-11-2.pdf}

\textbf{Exercise 9.8} or the United States GDP series (from
global\_economy):

\begin{itemize}
\tightlist
\item
  if necessary, find a suitable Box-Cox transformation for the data;
\item
  fit a suitable ARIMA model to the transformed data using ARIMA();
\item
  try some other plausible models by experimenting with the orders
  chosen;
\item
  choose what you think is the best model and check the residual
  diagnostics;
\item
  produce forecasts of your fitted model. Do the forecasts look
  reasonable?
\item
  compare the results with what you would obtain using ETS() (with no
  transformation).
\end{itemize}

\begin{Shaded}
\begin{Highlighting}[]
\NormalTok{usa\_gdp }\OtherTok{\textless{}{-}}\NormalTok{ global\_economy }\SpecialCharTok{|\textgreater{}} \FunctionTok{filter}\NormalTok{(Code }\SpecialCharTok{==} \StringTok{\textquotesingle{}USA\textquotesingle{}}\NormalTok{)}
\NormalTok{usa\_gdp }\SpecialCharTok{|\textgreater{}} \FunctionTok{gg\_tsdisplay}\NormalTok{(GDP, }\AttributeTok{plot\_type =} \StringTok{\textquotesingle{}partial\textquotesingle{}}\NormalTok{) }\SpecialCharTok{+} \FunctionTok{labs}\NormalTok{(}\AttributeTok{title =} \StringTok{"Non{-}transformed United States GDP"}\NormalTok{)}
\end{Highlighting}
\end{Shaded}

\includegraphics{Data-6224-Homework-6_files/figure-latex/unnamed-chunk-12-1.pdf}

\begin{itemize}
\item
  \begin{enumerate}
  \def\labelenumi{\alph{enumi}.}
  \tightlist
  \item
    if necessary, find a suitable Box-Cox transformation for the data;
  \end{enumerate}
\end{itemize}

ARIMA(1,1,0) with a drift was fitted to the tansformed data.

\begin{Shaded}
\begin{Highlighting}[]
\NormalTok{lambda  }\OtherTok{\textless{}{-}}\NormalTok{usa\_gdp }\SpecialCharTok{\%\textgreater{}\%}
  \FunctionTok{features}\NormalTok{(GDP, }\AttributeTok{features =}\NormalTok{ guerrero) }\SpecialCharTok{\%\textgreater{}\%}
  \FunctionTok{pull}\NormalTok{(lambda\_guerrero)}
\NormalTok{lambda}
\end{Highlighting}
\end{Shaded}

\begin{verbatim}
## [1] 0.2819443
\end{verbatim}

\begin{Shaded}
\begin{Highlighting}[]
\CommentTok{\# fit a suitable ARIMA model to the transformed data using ARIMA();}
\NormalTok{fit\_usa }\OtherTok{\textless{}{-}}\NormalTok{ global\_economy }\SpecialCharTok{\%\textgreater{}\%}
  \FunctionTok{filter}\NormalTok{(Code }\SpecialCharTok{==} \StringTok{"USA"}\NormalTok{) }\SpecialCharTok{\%\textgreater{}\%}
  \FunctionTok{model}\NormalTok{(}\FunctionTok{ARIMA}\NormalTok{(}\FunctionTok{box\_cox}\NormalTok{(GDP, lambda))) }

\FunctionTok{report}\NormalTok{(fit\_usa)}
\end{Highlighting}
\end{Shaded}

\begin{verbatim}
## Series: GDP 
## Model: ARIMA(1,1,0) w/ drift 
## Transformation: box_cox(GDP, lambda) 
## 
## Coefficients:
##          ar1  constant
##       0.4586  118.1822
## s.e.  0.1198    9.5047
## 
## sigma^2 estimated as 5479:  log likelihood=-325.32
## AIC=656.65   AICc=657.1   BIC=662.78
\end{verbatim}

\begin{Shaded}
\begin{Highlighting}[]
\CommentTok{\# transformed plot}
\NormalTok{usa\_gdp }\SpecialCharTok{\%\textgreater{}\%}
  \FunctionTok{gg\_tsdisplay}\NormalTok{(}\FunctionTok{box\_cox}\NormalTok{(GDP,lambda), }\AttributeTok{plot\_type=}\StringTok{\textquotesingle{}partial\textquotesingle{}}\NormalTok{) }\SpecialCharTok{+}
  \FunctionTok{labs}\NormalTok{(}\AttributeTok{title =} \StringTok{"Transformed United States GDP"}\NormalTok{)}
\end{Highlighting}
\end{Shaded}

\includegraphics{Data-6224-Homework-6_files/figure-latex/unnamed-chunk-13-1.pdf}

\begin{Shaded}
\begin{Highlighting}[]
\CommentTok{\# unit root test}
\NormalTok{usa\_gdp }\SpecialCharTok{\%\textgreater{}\%}
  \FunctionTok{features}\NormalTok{(}\FunctionTok{box\_cox}\NormalTok{(GDP,lambda), unitroot\_ndiffs) }
\end{Highlighting}
\end{Shaded}

\begin{verbatim}
## # A tibble: 1 x 2
##   Country       ndiffs
##   <fct>          <int>
## 1 United States      1
\end{verbatim}

\begin{Shaded}
\begin{Highlighting}[]
\CommentTok{\# modeling several }
\NormalTok{usa\_fit }\OtherTok{\textless{}{-}}\NormalTok{ global\_economy }\SpecialCharTok{\%\textgreater{}\%}
  \FunctionTok{filter}\NormalTok{(Code }\SpecialCharTok{==} \StringTok{"USA"}\NormalTok{) }\SpecialCharTok{\%\textgreater{}\%}
  \FunctionTok{model}\NormalTok{(}\AttributeTok{arima110 =} \FunctionTok{ARIMA}\NormalTok{(}\FunctionTok{box\_cox}\NormalTok{(GDP,lambda) }\SpecialCharTok{\textasciitilde{}} \FunctionTok{pdq}\NormalTok{(}\DecValTok{1}\NormalTok{,}\DecValTok{1}\NormalTok{,}\DecValTok{0}\NormalTok{)),}
        \AttributeTok{arima120 =} \FunctionTok{ARIMA}\NormalTok{(}\FunctionTok{box\_cox}\NormalTok{(GDP,lambda) }\SpecialCharTok{\textasciitilde{}} \FunctionTok{pdq}\NormalTok{(}\DecValTok{1}\NormalTok{,}\DecValTok{2}\NormalTok{,}\DecValTok{0}\NormalTok{)),}
        \AttributeTok{arima210 =} \FunctionTok{ARIMA}\NormalTok{(}\FunctionTok{box\_cox}\NormalTok{(GDP,lambda) }\SpecialCharTok{\textasciitilde{}} \FunctionTok{pdq}\NormalTok{(}\DecValTok{2}\NormalTok{,}\DecValTok{1}\NormalTok{,}\DecValTok{0}\NormalTok{)),}
        \AttributeTok{arima212 =} \FunctionTok{ARIMA}\NormalTok{(}\FunctionTok{box\_cox}\NormalTok{(GDP,lambda) }\SpecialCharTok{\textasciitilde{}} \FunctionTok{pdq}\NormalTok{(}\DecValTok{2}\NormalTok{,}\DecValTok{1}\NormalTok{,}\DecValTok{2}\NormalTok{)),}
        \AttributeTok{arima111 =} \FunctionTok{ARIMA}\NormalTok{(}\FunctionTok{box\_cox}\NormalTok{(GDP,lambda) }\SpecialCharTok{\textasciitilde{}} \FunctionTok{pdq}\NormalTok{(}\DecValTok{1}\NormalTok{,}\DecValTok{1}\NormalTok{,}\DecValTok{1}\NormalTok{)))}

\FunctionTok{glance}\NormalTok{(usa\_fit) }\SpecialCharTok{\%\textgreater{}\%} \FunctionTok{arrange}\NormalTok{(AICc) }\SpecialCharTok{\%\textgreater{}\%} \FunctionTok{select}\NormalTok{(.model}\SpecialCharTok{:}\NormalTok{BIC)}
\end{Highlighting}
\end{Shaded}

\begin{verbatim}
## # A tibble: 5 x 6
##   .model   sigma2 log_lik   AIC  AICc   BIC
##   <chr>     <dbl>   <dbl> <dbl> <dbl> <dbl>
## 1 arima120  6780.   -326.  656.  656.  660.
## 2 arima110  5479.   -325.  657.  657.  663.
## 3 arima111  5580.   -325.  659.  659.  667.
## 4 arima210  5580.   -325.  659.  659.  667.
## 5 arima212  5734.   -325.  662.  664.  674.
\end{verbatim}

\begin{Shaded}
\begin{Highlighting}[]
\CommentTok{\# d. choose what you think is the best model and check the residual diagnostics;}
\NormalTok{usa\_fit }\SpecialCharTok{\%\textgreater{}\%}
  \FunctionTok{select}\NormalTok{(arima120) }\SpecialCharTok{\%\textgreater{}\%}
  \FunctionTok{gg\_tsresiduals}\NormalTok{() }\SpecialCharTok{+}
  \FunctionTok{ggtitle}\NormalTok{(}\StringTok{"Residuals for ARIMA(1,2,0)"}\NormalTok{)}
\end{Highlighting}
\end{Shaded}

\includegraphics{Data-6224-Homework-6_files/figure-latex/unnamed-chunk-13-2.pdf}

\begin{Shaded}
\begin{Highlighting}[]
\CommentTok{\# e. produce forecasts of your fitted model. Do the forecasts look reasonable?}
\NormalTok{usa\_fit }\SpecialCharTok{\%\textgreater{}\%}
  \FunctionTok{forecast}\NormalTok{(}\AttributeTok{h=}\DecValTok{5}\NormalTok{) }\SpecialCharTok{\%\textgreater{}\%}
  \FunctionTok{filter}\NormalTok{(.model}\SpecialCharTok{==}\StringTok{\textquotesingle{}arima120\textquotesingle{}}\NormalTok{) }\SpecialCharTok{\%\textgreater{}\%}
  \FunctionTok{autoplot}\NormalTok{(global\_economy)}
\end{Highlighting}
\end{Shaded}

\includegraphics{Data-6224-Homework-6_files/figure-latex/unnamed-chunk-13-3.pdf}

\begin{Shaded}
\begin{Highlighting}[]
\CommentTok{\# f. compare the results with what you would obtain using ETS() (with no transformation).}
\NormalTok{fit\_ets }\OtherTok{\textless{}{-}}\NormalTok{ global\_economy }\SpecialCharTok{\%\textgreater{}\%}
  \FunctionTok{filter}\NormalTok{(Code }\SpecialCharTok{==} \StringTok{"USA"}\NormalTok{) }\SpecialCharTok{\%\textgreater{}\%}
  \FunctionTok{model}\NormalTok{(}\FunctionTok{ETS}\NormalTok{(GDP))}

\FunctionTok{report}\NormalTok{(fit\_ets)}
\end{Highlighting}
\end{Shaded}

\begin{verbatim}
## Series: GDP 
## Model: ETS(M,A,N) 
##   Smoothing parameters:
##     alpha = 0.9990876 
##     beta  = 0.5011949 
## 
##   Initial states:
##          l[0]        b[0]
##  448093333334 64917355687
## 
##   sigma^2:  7e-04
## 
##      AIC     AICc      BIC 
## 3190.787 3191.941 3201.089
\end{verbatim}

\begin{Shaded}
\begin{Highlighting}[]
\NormalTok{fit\_ets }\SpecialCharTok{\%\textgreater{}\%}
  \FunctionTok{forecast}\NormalTok{(}\AttributeTok{h=}\DecValTok{5}\NormalTok{) }\SpecialCharTok{\%\textgreater{}\%}
  \FunctionTok{autoplot}\NormalTok{(global\_economy)}
\end{Highlighting}
\end{Shaded}

\includegraphics{Data-6224-Homework-6_files/figure-latex/unnamed-chunk-13-4.pdf}

The analysis of the United States GDP series from the
\texttt{global\_economy} dataset involved applying a Box-Cox
transformation to stabilize the variance, followed by fitting an
ARIMA(1,2,0) model. The transformed series showed a clear upward trend
and strong autocorrelation, indicating the need for differencing. After
fitting the ARIMA model, the residual diagnostics suggested that the
residuals behaved like white noise, with no significant autocorrelation
and an approximately normal distribution. Forecasts from the ARIMA model
were reasonable, showing a smooth upward trajectory with appropriate
uncertainty bounds. A comparison with the ETS model (without
transformation) produced similar forecast behavior, suggesting that both
approaches captured the underlying trend effectively. However, ARIMA may
offer better interpretability due to its explicit handling of
differencing and autocorrelation.

\end{document}
